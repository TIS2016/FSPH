\documentclass[12pt,a4paper]{report}
\usepackage[top=2.5cm,bottom=2.5cm,left=1.5cm,right=1.5cm]{geometry}

\usepackage{cmap}
\usepackage{graphicx}
\usepackage[english,slovak]{babel}
\usepackage[utf8]{inputenc}
\usepackage[T1]{fontenc}
%\usepackage{lmodern}

\usepackage{amssymb}
\usepackage{amsmath}

\usepackage{amsthm}
\newtheorem{veta}{Veta}[section]
\newtheorem{lema}[veta]{Lema}
\theoremstyle{definition}
\newtheorem{definicia}{Definícia}[chapter]
\theoremstyle{remark}
\newtheorem*{poznamka}{Poznámka}

\usepackage{setspace}
\onehalfspacing

\begin{document}

%\includepdf{zadanie.pdf}

%%% DEFINÍCIE NÁZVOV
\def\nazov{ANALÝZA TECHNOLÓGIÍ}
\def\autorJ{Jaroslav Fúska }
\def\autorT{Tomáš Sláma }
\def\autorH{ Martin Heinz }
\def\autorM{Michal Puškel }
\def\fakulta{Fakulta matematiky, fyziky a~informatiky}
\def\univerzita{Univerzita Komenského v~Bratislave}
\def\mesto{Bratislava}
\def\typprace{Športový klub}
\def\rok{2016}
%%% OBAL
\thispagestyle{empty}
\begin{center}
\textsc{\LARGE\univerzita}\\
\bigskip\textsc{\LARGE\fakulta}\\
\vfill\textsc{\Huge\nazov}\\
\medskip{\Large\typprace}\\
\vfill{\large\rok\hfill\autorJ\\ \hfill\autorT \\ \hfill \autorH \\ \hfill \autorM}
\end{center}

\tableofcontents

\chapter{Technológie}
%\setcounter{page}{5}
\addcontentsline{toc}{chapter}{\numberline {}Technológie}

\section{HTML}
\begin{itemize}
\item HTML (Hypertext Markup Language) bude použité na základnú štruktúru webovej aplikácie a rozmiestnenie objektov.
\item využitá bude verzia HTML 5.0
\item HTML bude kombinované s viacerými jazykmi, ako napríklad PHP, CSS...
\end{itemize}

\section{CSS}
\begin{itemize}
\item CSS (Cascading Style Sheets) bude použité na vytvorenie, úpravu dizajnu a štylizovanie webovej aplikácie.
\item využitá bude verzia CSS 3.0
\item CSS bude využité aj na vytvorenie responzivity webovej aplikácie
\end{itemize}

\section{Bootstrap}
\begin{itemize}
\item Bootstrap je jednoduchá voľne stiahnuteľná sada nástrojov na tvorbu webových aplikácií
\item obsahuje návrhárske šablóny založené na HTML a CSS
\item pre využívanie je potrebná dobrá znalosť HTML a CSS
\item využíva LESS deklarácie, čo umožňuje používanie napr. premenných a funkcií priamo v CSS kóde
\item dôvod použitia: efektívnejšia práca pri navrhovaní a vytváraní responzívnej aplikácie
\end{itemize}

\section{LESS}
\begin{itemize}
\item LESS je "dynamic style sheet"  jazyk, ktorý je možné skompilovať do CSS
\item umožňuje definovať a používať premenné a funkcie
\item dôvod použitia: efektívnejšia práca pri úpravách CSS (napr. zmena farieb...)
\end{itemize}

\section{PHP}
\begin{itemize}
\item PHP je skriptovací jazyk, ktorý sa používa najmä na programovanie klient-server aplikácií
\item dôvod použitia: PHP dokáže spolupracovať s relačnými databázami, ako napr. MySQL alebo SQLite, ktoré budú súčasťou nášho projektu
\item predpokladané použitie verzie PHP 5.5 / 7
\end{itemize}

\section{JavaScript}
\begin{itemize}
\item JS je multiplatformový, objektovo orientovaný skriptovací jazyk na strane klienta
\item dôvod použitia: využitie na interakciu s používateľom, prácu s dátami z GoogleMaps api 
\end{itemize}

\section{Laravel}
\begin{itemize}
\item open source framework pre PHP
\item Laravel zahŕňa ako základné funkcie:
	
	\begin{itemize}
	\item autentifikáciu - kontrolu prístupu
	\item routovanie - spravovanie, smerovanie a spracovanie dotazov na jednom mieste
	\item databázu - všetky nástroje potrebné na komunikáciu s databázou
	\item mail - posielanie emailov s prílohami a vloženými súbormi
	\item sessios
	\item caching 
	\end{itemize}

\item dôvod použitia: zjednodušenie práce pri programovaní funkcií, bezpečnejšie a efektívnejšie algoritmy na autentifikáciu a komunikáciu s databázou
\end{itemize}

\section{SQL}
\begin{itemize}
\item štruktúrovaný vzhľadávací jazyk určený na manipuláciu (výber, vkladanie, úpravu a mazanie) a definúciu údajov v relačných databázových systémoch
\item dôvod použitia: jedna z hlavných súčastí projektu budú relačné databázy a práca s nimi
\end{itemize}

\section{SQLite}
\begin{itemize}
\item je relačná databáza pracujúca bez servera, čiže používa iba klientsku časť
\item nie je potrebná jej inštalácia ani žiadne konfiguračné súbory
\item jednoduchý import a export tabuliek
\item dôvod použitia: využitie na uloženie dát potrebných pre fungovanie webovej aplikácie
\end{itemize}

\section{WAMP / LAMP}
\begin{itemize}
\item apache, MySQL a PHP aplikačná serverová platforma
\item dôvod použitia: využitie počas testovania webovej aplikácie bez prístupu k serveru na ktorom bude výsledná aplikácia bežať
\end{itemize}



\end{document} 